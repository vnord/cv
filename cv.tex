% xelatex
\documentclass[letterpaper,
		%twocolumn,
		10pt]{article}
\usepackage[utf8]{inputenc}
\usepackage{metalogo}
\usepackage{xifthen}
\usepackage[colorlinks=true,urlcolor=Blue]{hyperref}
\usepackage{graphicx}
\usepackage{fontspec}
\usepackage[T1]{fontenc}
\usepackage[dvipsnames]{xcolor}
\usepackage{titlesec}
\usepackage[margin=1in]{geometry}
\usepackage{titling}
\usepackage{libertine}

\usepackage{fontspec}
\usepackage{fontawesome}

% Macro to allow image links in XeLaTeX
\ifxetex
  \usepackage{letltxmacro}
  \setlength{\XeTeXLinkMargin}{1pt}
  \LetLtxMacro\SavedIncludeGraphics\includegraphics
  \def\includegraphics#1#{% #1 catches optional stuff (star/opt. arg.)
    \IncludeGraphicsAux{#1}%
  }%
  \newcommand*{\IncludeGraphicsAux}[2]{%
    \XeTeXLinkBox{%
      \SavedIncludeGraphics#1{#2}%
    }%
  }%
\fi
%%%%%%%

% Bold contents of a link
\let\oldhref\href
\renewcommand{\href}[3][blue]{\oldhref{#2}{\color{#1}{#3}}}

% Your name goes here:
\author{Ari von Nordenskjöld}

% Update date set to last compile:
\date{\today}

% Custom title command.
\renewcommand{\maketitle}{
	\hspace{.25\textwidth}
	\begin{minipage}[t]{.5\textwidth}
\par{\centering{\LARGE  \bfseries{\theauthor}}\par}
	\end{minipage}
	\begin{minipage}[t]{.25\textwidth}
{\footnotesize\hfill{}\color{gray}
\hfill{}Download this document:

\hfill{}\href[gray]{http://vnord.net/cv.pdf}{http://vnord.net/cv.pdf \faDownload}

\hfill{}(Last updated \thedate.)
}
	\end{minipage}
}



% Setting the font I want:
\renewcommand{\familydefault}{\sfdefault}
\usepackage{sqrcaps}

% Making the \entry command
\newcommand{\entry}[4]{
\ifthenelse{\isempty{#3}}
{\slimentry{#1}{#2}}{

\begin{minipage}[t]{.15\linewidth}
\hfill \textsc{#1}
\end{minipage}
\hfill\vline\hfill
\begin{minipage}[t]{.80\linewidth}
{\bf#2}\\\textit{#3} \small{#4}
\end{minipage}\\
\vspace{.2cm}
}}

\newcommand{\slimentry}[2]{
\begin{minipage}[t]{.15\linewidth}
\hfill \textsc{#1}
\end{minipage}
\hfill\vline\hfill
\begin{minipage}[t]{.80\linewidth}
#2
\end{minipage}\\
\vspace{.25cm}
}

\newcommand{\sentry}[2]{
\begin{minipage}[t]{.15\linewidth}
\hfill \textsc{#1}
\end{minipage}
\hfill\vline\hfill
\begin{minipage}[t]{.80\linewidth}
#2
\end{minipage}\\
\vspace{-.15cm}
}% end \entry command definition

% Some macros because I'm lazy:
\newcommand{\ch}{Chalmers University of Technology}

% Macros for people's names including link to their websites
% \newcommand{\tgb}{\href{http://coglanglab.com}{Tom Bever}}

\let\lineheight\baselineskip

% Link images
% \newcommand{\pdf}{\includegraphics[height=.85em]{cv/pdf.png}}
% \newcommand{\yt}{\includegraphics[height=.85em]{cv/yt.png}}
% \newcommand{\gh}{\includegraphics[height=.85em]{cv/gh.png}}
% \newcommand{\www}{\includegraphics[height=.85em]{cv/www.png}}
% \newcommand{\email}{\includegraphics[height=.85em]{cv/email.png}}

% Custom section spacing and formatting
\titleformat{\part}{\Huge\scshape\filcenter}{}{1em}{}
\titleformat{\section}{\Large\bf\raggedright}{}{1em}{}[{\titlerule[2pt]}]
\titlespacing{\section}{0pt}{3pt}{7pt}
\titleformat{\subsection}{\large\bfseries\centering}{}{0em}{\underline}%[\rule{3cm}{.2pt}]
\titlespacing{\subsection}{0pt}{7pt}{7pt}

% No indentation
\setlength{\parindent}{0in}

\begin{document}

\maketitle

\section{Basic Info}

\begin{minipage}[t]{.4\linewidth}
\begin{tabular}{cp{.75\linewidth}}
	\baselineskip=20pt
    \faEnvelope & \href{mailto:ari@vnord.net}{ari@vnord.net}\\
    \faGlobe & \href{http://vnord.net}{http://vnord.net}
\end{tabular}
\end{minipage}
\begin{minipage}[t]{.4\linewidth}
\begin{tabular}{rl}
	\faGithub & \href{http://github.com/vnord}{github.com/vnord} \\
    \faPhone & +46 703716960
\end{tabular}
\end{minipage}
\begin{minipage}[t]{\linewidth}
\begin{tabular}{rl}
    & Älvsborgsplan 6 \\
    & 414 71 Göteborg
\end{tabular}
\end{minipage}

\vspace{0.25cm}
% \begin{itemize}
% \item Fourth-year Computer Science and Engineering student
% \item Maintainer and developer of an Arch Linux-based metadistribution at \href{https://larbs.xyz}{https://larbs.xyz \www}.
% \item Doctoral student in linguistics at the \href{https://linguistics.arizona.edu}{University of Arizona \www}, focusing mostly on deriving linguistic traits and tendencies from non-linguistic externals and theoretical reform.

% For more information, go here: \href{https://lukesmith.xyz/linguistics}{https://lukesmith.xyz/linguistics \www}.
% \item Since late 2016, I run a \href{https://youtube.com/c/lukesmithxyz}{YouTube channel \yt} mostly on technology tutorials and Linux configuration and system management.
% \end{itemize}

\section{Education}

\entry{2018-2020}
    {MSc. in Computer Science - Algorithms, Languages and Logic}
    {\ch, Gothenburg}
    {\\Transcript from March 19, 2019 available
    at \href[gray]{http://vnord.net/transcript.pdf}{http://vnord.net/transcript.pdf \faDownload}}

\entry{2015-2018}
    {BSc. in Computer Engineering}
    {\ch, Gothenburg}
    {\\\textit{Thesis title:} Port Call Synchronization \textit{Thesis company:} RISE Viktoria AB, Gothenburg
     \\\textit{Thesis subject:} Automating and optimising the generation of recommended times of arrival for ships coming in to
    ports by using data analysis and machine learning}

\entry{2014}
    {Foundation Year}
    {\ch, Gothenburg}
    {\\\textit{Subjects:} Calculus, Physics, and Chemistry}


\section{Experience}

\entry{Summer 2018}
    {Back-end Developer}
    {Rise Viktoria AB, Gothenburg}
    {\\Developed various back-end features for the PortCDM information sharing platform, including an extensive
    notification system.
    \textit{Skills:} Java EE, WildFly Application Server, Linux}

\entry{2017-2018}
    {Project Manager for initiative to help specially talented children}
    {The non-profit association Intize, Gothenburg}
    {\\Recruited mentors as well as specially talented children, and coordinated mentor groups}

\entry{Fall 2016}
    {Supplemental Instruction Leader}
    {\ch, Gothenburg}
    {\\Coached groups of students taking a course in Discrete Mathematics}

\entry{Fall 2016}
    {Mentor for Individual Student}
    {\ch, Gothenburg}
    {\\Supported and helped a student with special needs manage their studies}

\section{Volunteering}

\entry{2017 - 2018}
    {Mentor for Specially Talented Children}
    {The non-profit association Intize, Gothenburg}
    {\\Made maths more fun and challenging for a group of three specially talented kids}

\entry{2016 - 2017}
    {Mentor for High School Students}
    {The non-profit association Intize, Gothenburg}
    {\\Helped a group of five high school students with their math homework for two hours every week}

\entry{Summer 2015}
    {WWOOF Volunteer}
    {Alpes-Mairitmes, France}
    {\\Helped out at a farm, milked cows, and made cheese}

\section{Tools I Use}

\subsection{Workflow}

I only use GNU/Linux-based operating systems such as \textbf{Arch Linux} or \textbf{Gentoo Linux} on my personal systems,
though I am also comfortable with other operating systems such as MacOS and Windows.
I usually use a tiling window manager (\textbf{i3-gaps}), and program and edit files with \textbf{Neovim}.
I compile most of my documents using \textbf{Pandoc}, but sometimes I use a more standard \LaTeX \, setup with \textbf{Biber}
for references.

\subsection{Programming Languages}

I am not exceedingly skilled when it comes to using any specific programming language (except perhaps Haskell), but
rather try to use the right language for whatever I am trying to accomplish, and am very happy
to learn new languages and paradigms. I really like to use functional principles regardless of what
language I am working in. The languages that I have
the most experience with are \textbf{Haskell}, \textbf{C}, and \textbf{Java}. I have also tinkered
with \textbf{Erlang}, \textbf{Agda}, \textbf{Idris}, \textbf{Scala}, \textbf{C++},
\textbf{Python}, \textbf{Rust}, \textbf{LISP}, \textbf{JavaScript}, \textbf{PureScript}, and some assembly languages. I am
confident in using and designing relational databases using database management systems such as
\textbf{PostgreSQL} and \textbf{MySQL}. During my studies I have also used
\textbf{VHDL}, \textbf{MatLab}, and \textbf{R}.

\section{Languages}

\sentry{English}{Fluent in writing and speech at an academic level}

\sentry{Swedish}{Native speaker}

\sentry{French}{Capable of communicating at an intermediate level, approximately B2 in the CEFR}

\sentry{German}{Capable of communicating at an intermediate level, approximately B2 in the CEFR}

\section{Hobbies}

I read a lot - everything from classics and science fiction, to history, economics, sociology, and poetry.
I try to read as much as I can in the language that it was originally written, and I am working
on learning Latin. I've played the violin since I was very young, and when I get more time
I will take up olympic fencing again.

% Classical languages, human evolution and prehistory, free (i.e. libre) software, survivalism, cybernetics, medieval thought, Rhaeto-Romance poetry. I run a web forum at \href{https://forum.lukesmith.xyz}{forum.lukesmith.xyz}.

\section{References}
    
Letters of recommendation from Chalmers University of Technology as well as Intize are available upon request.
If you are looking for someone to vouch for my personal and/or professional character, that can also be arranged.
Just send me an email.


\end{document}
